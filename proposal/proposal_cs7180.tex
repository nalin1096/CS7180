\documentclass{article}

% if you need to pass options to natbib, use, e.g.:
%     \PassOptionsToPackage{numbers, compress}{natbib}
% before loading neurips_2018

% ready for submission
% \usepackage{neurips_2018}

% to compile a preprint version, e.g., for submission to arXiv, add add the
% [preprint] option:
%     \usepackage[preprint]{neurips_2018}

% to compile a camera-ready version, add the [final] option, e.g.:
     \usepackage[final]{nips_2018}

% to avoid loading the natbib package, add option nonatbib:
%     \usepackage[nonatbib]{neurips_2018}

\usepackage[utf8]{inputenc} % allow utf-8 input
\usepackage[T1]{fontenc}    % use 8-bit T1 fonts
\usepackage{hyperref}       % hyperlinks
\usepackage{url}            % simple URL typesetting
\usepackage{booktabs}       % professional-quality tables
\usepackage{amsfonts}       % blackboard math symbols
\usepackage{nicefrac}       % compact symbols for 1/2, etc.
\usepackage{microtype}      % microtypography

\title{CS 7180 Project Proposal}

% The \author macro works with any number of authors. There are two commands
% used to separate the names and addresses of multiple authors: \And and \AND.
%
% Using \And between authors leaves it to LaTeX to determine where to break the
% lines. Using \AND forces a line break at that point. So, if LaTeX puts 3 of 4
% authors names on the first line, and the last on the second line, try using
% \AND instead of \And before the third author name.

\author{%
  Nalin Gupta \\
  Department of Computer Science\\
  Northeastern University\\
  Boston, MA 02115 \\
  \texttt{gupta.nal@husky.neu.edu} \\
  \And
  Christopher Botica\\
  Department of Computer Science\\
  Northeastern University\\
  Boston, MA 02115 \\
  \texttt{botica.c@husky.neu.edu} \\
  \And
  Tyler Brown\\
  Department of Computer Science\\
  Northeastern University\\
  Boston, MA 02115 \\
  \texttt{brown.tyler@husky.neu.edu} \\
  % Coauthor \\
  % Affiliation \\
  % Address \\
  % \texttt{email} \\
  % \AND
  % Coauthor \\
  % Affiliation \\
  % Address \\
  % \texttt{email} \\
  % \And
  % Coauthor \\
  % Affiliation \\
  % Address \\
  % \texttt{email} \\
  % \And
  % Coauthor \\
  % Affiliation \\
  % Address \\
  % \texttt{email} \\
}

\begin{document}
% \nipsfinalcopy is no longer used

\maketitle

\begin{abstract}
  We might see how denoising algorithms perform under different lighting
  conditions. We're very interested in solving a problem with practical
  applications. 
\end{abstract}

\section{Introduction}

Most likely want to reference ideas from this paper if we're
sticking with denoising \cite{tian2018deep}.

\begin{itemize}
  \item describe what is the problem you want to solve
  \item provide examples where solving your problem is highly valuable 
  \item cite other papers that have tried to solve a similar problem
    (use google scholar to search)
  \item explain why the solutions from existing papers are still not enough
\end{itemize}

\section{Proposed Idea}

\begin{itemize}
\item describe your idea to solve the problem
\item explain the novelty of your idea, compared with other papers
\item what datasets/baselines are you going to use
\item how do you measure success
\end{itemize}

\textit{``...we highly encourage bold and original ideas.''}


\bibliographystyle{unsrt}
\bibliography{references}

\end{document}
