\documentclass{article}

% if you need to pass options to natbib, use, e.g.:
%     \PassOptionsToPackage{numbers, compress}{natbib}
% before loading neurips_2018

% ready for submission
% \usepackage{neurips_2018}

% to compile a preprint version, e.g., for submission to arXiv, add add the
% [preprint] option:
%     \usepackage[preprint]{neurips_2018}

% to compile a camera-ready version, add the [final] option, e.g.:
     \usepackage[final]{nips_2018}

% to avoid loading the natbib package, add option nonatbib:
%     \usepackage[nonatbib]{neurips_2018}

\usepackage[utf8]{inputenc} % allow utf-8 input
\usepackage[T1]{fontenc}    % use 8-bit T1 fonts
\usepackage{hyperref}       % hyperlinks
\usepackage{url}            % simple URL typesetting
\usepackage{booktabs}       % professional-quality tables
\usepackage{amsfonts}       % blackboard math symbols
\usepackage{nicefrac}       % compact symbols for 1/2, etc.
\usepackage{microtype}      % microtypography
\usepackage{graphicx}
\usepackage{subcaption}

\graphicspath{ {imgs/} }

\title{CS 7180 Milestone 2}

% The \author macro works with any number of authors. There are two commands
% used to separate the names and addresses of multiple authors: \And and \AND.
%
% Using \And between authors leaves it to LaTeX to determine where to break the
% lines. Using \AND forces a line break at that point. So, if LaTeX puts 3 of 4
% authors names on the first line, and the last on the second line, try using
% \AND instead of \And before the third author name.

\author{%
  Nalin Gupta \\
  \texttt{NUID: 001726854} \\
  \And
  Christopher Botica\\
  \texttt{NUID: 001726854} \\
  \And
  Tyler Brown\\
  \texttt{NUID: 001684955} \\
  % Coauthor \\
  % Affiliation \\
  % Address \\
  % \texttt{email} \\
  % \AND
  % Coauthor \\
  % Affiliation \\
  % Address \\
  % \texttt{email} \\
  % \And
  % Coauthor \\
  % Affiliation \\
  % Address \\
  % \texttt{email} \\
  % \And
  % Coauthor \\
  % Affiliation \\
  % Address \\
  % \texttt{email} \\
}

\begin{document}
% \nipsfinalcopy is no longer used

\maketitle

\section{Introduction}

We are trying to make it more cost effective for AI to see in the dark. Chen et. al.
(2018) \cite{chen2018learning} successfully demonstrated that a neural network could
process dark images into light images. Their model requires a specially curated dataset
by a professional photographer, and substantial processing resources. Our group is trying to
replicate Chen et. al. (2018) functionality using data augmentation on existing datasets.
Our goal is to make the Chen et. al. (2018) more cost efficient and generalizable across
domains.

The novelty of our approach stems from the idea of ``more for less''. Our
model drastically reduces the overhead costs of data collection by
synthesizing readily available training data (MIT-Adobe FiveK). This is
particularly beneficial in domains where collecting images pairs is
expensive/time consuming.

This is a hard problem because simulating natural phenomena using augmented data
is not easy. In other words, ``[i]mitation is possible because distinct physical
systems can be organized to exhibit nearly identical behavior'' \cite{Simon:1996:SA:237774}.
Identifying this identical behavior using statistical distributions of image properties
such as distortion, and lighting.

\section{Related Work}

In the past, the problem of enhancing low light images has been tackled via
noise reduction. This noise becomes dominant especially in low-light images
due to low SNR. Remez et. al. proposed a deep CNN for noise reduction under
the assumption that this low-light noise belongs to a Poisson
distribution \cite{remez2017deep}.  They used images from ImageNet
\cite{imagenet_cvpr09} as their ground truth data
and added synthetic Poisson noise to simulate corrupted images. Even though
their model outperform the state-of-the art de-noiser ``BM3D'', it does not
scale well to real world images, due to their underlying assumptions.
Furthermore, their model only denoises images but does not brighten them.
Motivated by these downfalls, Chen et. al., proposed an end-to-end CNN,
``See-in-the-Dark'' (SID), which brightens extremely low light images and
removes noise without making any underlying assumptions
\cite{chen2018learning}. However these advances come with the added expense
of collecting large amounts of low
light and bright light images. In the absence of a true vs noisy image
dataset, the team captured scenes using various exposure times to generate
true (bright light) and corrupted (low light) image pairs called
``See-in-the-Dark Dataset'' (SID Dataset \footnote{https://github.com/cchen156/Learning-to-See-in-the-Dark}). Furthermore, their model is camera
specific and not easily generalizable.\newline

\section{Method/Model}

\section{Experiment}

\section{Discussion}


\bibliographystyle{unsrt}
\bibliography{references}

\end{document}
